% prologue.tex
%
% Author       : James Mnatzaganian
% Contact      : http://techtorials.me
% Date Created : 08/27/15
%
% Description  : The prologue used by "thesis.tex".
%
% Copyright (c) 2015 James Mnatzaganian

% NOTE: All filler text has "TODO" written. This must be removed in the final copy!

% Initialize starting pages to use Roman numerals
\frontmatter

% \begin{acknowledgments}
%%%%%%%%%%%%%%%%%%%%%%%%%%%%%%%%%%%%%%%%%%%%%%%%%%%%%%%%%%%%%%%%%%%%%%%%%%%%%%%
% Acknowledgments
%%%%%%%%%%%%%%%%%%%%%%%%%%%%%%%%%%%%%%%%%%%%%%%%%%%%%%%%%%%%%%%%%%%%%%%%%%%%%%%

\begin{acknowledgments}
	\vfill
	\begin{center}
		\indent Most importantly, thank you to Dr. S. Jay Yang for providing me with the opportunity to assist in research throughout my undergraduate study. Your continued feedback and encouragement to challenge myself over the years has helped me to grow into a strong engineer and critical thinker. You have had a profound impact on me, and I can never thank you enough for that. Next, I would like to thank Dr. Raymond Ptucha, Dr. Sonia Lopez-Alarcon, and the rest of the RIT Computer Engineering faculty for their willingness to educate the next generation of Computer Engineers. Finally, I would like to thank (the soon to be Dr.) Stephen Moskall who provided me my first opportunity to research with him and Dr. Yang when I was in my second year of undergrad. All the advice you have given me has helped me to grow substantially. 
	\end{center}
	\vfill
\end{acknowledgments}
% \end{acknowledgments}

% \begin{dedication}
%%%%%%%%%%%%%%%%%%%%%%%%%%%%%%%%%%%%%%%%%%%%%%%%%%%%%%%%%%%%%%%%%%%%%%%%%%%%%%%
% Dedication
%%%%%%%%%%%%%%%%%%%%%%%%%%%%%%%%%%%%%%%%%%%%%%%%%%%%%%%%%%%%%%%%%%%%%%%%%%%%%%%

\begin{dedication}
	\vfill
	\begin{center}
		To my mother and father, Debbie and Jack Sweet, who taught me the value of a strong work ethic. And to my brother, John, who always set an extremely high bar for me to try and beat. Without your continued support, guidance, and devotion, I never would have been able to achieve all that I have. 
	\end{center}
	\vfill
\end{dedication}
% \end{dedication}

%\begin{abstract}
%%%%%%%%%%%%%%%%%%%%%%%%%%%%%%%%%%%%%%%%%%%%%%%%%%%%%%%%%%%%%%%%%%%%%%%%%%%%%%%
% Abstract
%%%%%%%%%%%%%%%%%%%%%%%%%%%%%%%%%%%%%%%%%%%%%%%%%%%%%%%%%%%%%%%%%%%%%%%%%%%%%%%

% Abstract
\begin{abstract}
	
	Cyber attacks perpetrated against enterprise computer networks continue to grow in number, severity, and complexity as our reliance on such networks grows. Despite this, proactive cyber security remains an open challenge as cyber alert data lacks homogeneity, is inherently imbalanced, and difficult to isolate and apply labels to. Currently, there is no commonly accepted way to generate synthetic alert data. Furthermore, there are no metrics to quantify the fidelity of synthetically generated alerts or identify critical attributes within the data.
	
	Generative Adversarial Networks are a Deep Learning architecture which have been used extensively in the fields of Computer Vision and Natural Language Processing to synthetically generate data. These models have been shown to effectively learn the distribution of data from the ground truth. Additionally, to create realistic data it is posited that these models learn pertinent interactions between output features. 
	
	This work explores the usage of Generative Adversarial Networks trained on cyber alert data with the intent of synthetically generating alerts and revealing critical feature interactions. Unique preprocessing steps, model considerations, and training methods are reviewed. Additionally, a list of criteria defining desirable attributes for cyber alert data metrics is provided. Several metrics which meet these criteria are proposed and used to identify the fidelity and critical relationships of synthetically generated alert data. Finally, through these metrics, we show that adding a mutual information constraint to GANs increases the quality of outputs and decreases output mode dropping. 
	
\end{abstract}

%%%%%%%%%%%%%%%%%%%%%%%%%%%%%%%%%%%%%%%%%%%%%%%%%%%%%%%%%%%%%%%%%%%%%%%%%%%%%%%
% Introductory Lists and Tables
%%%%%%%%%%%%%%%%%%%%%%%%%%%%%%%%%%%%%%%%%%%%%%%%%%%%%%%%%%%%%%%%%%%%%%%%%%%%%%%

% Add TOC, list of figures, list of tables in that order
\makealllists

% Add the acronyms
\glsaddall
\printglossary[type=\acronymtype]


% Reset all acronyms
\glsresetall

% Start using Arabic numbers
\mainmatter
% \end{introductory lists and tabels}