\frontmatter
% \begin{acknowledgments}
%%%%%%%%%%%%%%%%%%%%%%%%%%%%%%%%%%%%%%%%%%%%%%%%%%%%%%%%%%%%%%%%%%%%%%%%%%%%%%%
% Acknowledgments
%%%%%%%%%%%%%%%%%%%%%%%%%%%%%%%%%%%%%%%%%%%%%%%%%%%%%%%%%%%%%%%%%%%%%%%%%%%%%%%

\begin{acknowledgments}
	\vfill
	\begin{center}
		\indent Most importantly, thank you to Dr. S. Jay Yang for providing me with the opportunity to assist in research throughout my undergraduate study. Your continued feedback and encouragement to challenge myself over the years has helped me to grow into a strong engineer and critical thinker. You have had a profound impact on me, and I can never thank you enough for that. Next, I would like to thank Dr. Raymond Ptucha, Dr. Sonia Lopez-Alarcon, and the rest of the RIT Computer Engineering faculty for their willingness to educate the next generation of Computer Engineers. Finally, I would like to thank (the soon to be Dr.) Stephen Moskal who provided me my first opportunity to research with him and Dr. Yang when I was in my second year of undergrad. The advice, feedback, and good times over the last few years have been invaluable. 
	\end{center}
	\vfill
\end{acknowledgments}
% \end{acknowledgments}

% \begin{dedication}
%%%%%%%%%%%%%%%%%%%%%%%%%%%%%%%%%%%%%%%%%%%%%%%%%%%%%%%%%%%%%%%%%%%%%%%%%%%%%%%
% Dedication
%%%%%%%%%%%%%%%%%%%%%%%%%%%%%%%%%%%%%%%%%%%%%%%%%%%%%%%%%%%%%%%%%%%%%%%%%%%%%%%

\begin{dedication}
	\vfill
	\begin{center}
		To my mother and father, Debbie and Jack Sweet, who taught me the value of a strong work ethic. And to my brother, John, who always set an extremely high bar for me to try and beat. Without your continued support, guidance, and devotion, I never would have been able to achieve all that I have. 
	\end{center}
	\vfill
\end{dedication}
% \end{dedication}

%\begin{abstract}
%%%%%%%%%%%%%%%%%%%%%%%%%%%%%%%%%%%%%%%%%%%%%%%%%%%%%%%%%%%%%%%%%%%%%%%%%%%%%%%
% Abstract
%%%%%%%%%%%%%%%%%%%%%%%%%%%%%%%%%%%%%%%%%%%%%%%%%%%%%%%%%%%%%%%%%%%%%%%%%%%%%%%

% Abstract
\begin{abstract}
	% Focus on the challenge of creating cyber alert data in abstract
	% Second paragraph is more about methods and uniqueness
	
	Cyber attacks infiltrating enterprise computer networks continue to grow in number, severity, and complexity as our reliance on such networks grows. Despite this, proactive cyber security remains an open challenge as cyber alert data is often not available for study. Furthermore, the data that is available is stochastically distributed, imbalanced, lacks homogeneity, and relies on complex interactions with latents aspects of the network structure. Currently, there is no commonly accepted way to model and generate synthetic alert data for further study; there are also no metrics to quantify the fidelity of synthetically generated alerts or identify critical attributes within the data.
	
	This work proposes solutions to both the modeling of cyber alerts and how to score the fidelity of such models. Generative Adversarial Networks are employed to generate cyber alert data taken from two collegiate penetration testing competitions. A list of criteria defining desirable attributes for cyber alert data metrics is provided. Several statistical and information-theoretic metrics, such as histogram intersection and conditional entropy, meet these criteria and are used for analysis. Using these metrics, critical relationships of synthetically generated alerts may be identified and compared to data from the ground truth distribution. Finally, through these metrics, we show that adding a mutual information constraint to the model's generation increases the quality of outputs and successfully captures alerts that occur with low probability.
	
\end{abstract}

%%%%%%%%%%%%%%%%%%%%%%%%%%%%%%%%%%%%%%%%%%%%%%%%%%%%%%%%%%%%%%%%%%%%%%%%%%%%%%%
% Introductory Lists and Tables
%%%%%%%%%%%%%%%%%%%%%%%%%%%%%%%%%%%%%%%%%%%%%%%%%%%%%%%%%%%%%%%%%%%%%%%%%%%%%%%

% Add TOC, list of figures, list of tables in that order
\makealllists

% Add the acronyms
\glsaddall
\printglossary[type=\acronymtype]


% Reset all acronyms
\glsresetall

% Start using Arabic numbers
\mainmatter
% \end{introductory lists and tabels}